\renewcommand{\abstractname}{前言}
\begin{abstract}
    分布式系统是位于联网计算机上的组件仅通过传递消息来通信和协调其动作的系统。 此定义导致分布式系统具有以下特别重要的特征:组件并发、缺少全局时钟以及组件独立故障\cite{coulouris2005distributed}。群集由并行执行实际处理的多台“工作”计算机组成。这些工作程序节点通常对直接用户访问是隐藏的。用户登录到提供系统管理的“头节点”计算机,在头节点上提交并行处理任务,然后该并行处理任务将并行任务分配给某些(或所有)工作程序节点。用户提供诸如需要多少工作节点之类的规范。为了允许多个用户以有组织的方式使用集群,可以使用开源集群管理软件\cite{6871791}。Ray实现了一个统一的界面,该界面可以表示任务并行和基于参与者的计算,并由单个动态执行引擎支持。为了满足性能要求,Ray使用了分布式调度程序和分布式且容错的存储来管理系统的控制状态。Ray将任务并行和参与者编程模型统一在一个单一的动态任务图中,并采用了一个可伸缩的架构,该架构由全局控制存储和自底向上的分布式调度程序支持。这种体系结构同时实现的编程灵活性、高吞吐量和低延迟对于新兴的人工智能工作负载特别重要,这些工作负载产生的任务在资源需求、持续时间和功能上各不相同\cite{moritz2018ray}。

    动态事件树风险分析软件(Dynamic Event Tree Risk-Informed Analysis,DETRIA)是一个为核电厂确定论与概率论耦合安全分析的软件。该软件在运行时需调用RELAP5为其计算(RELAP5仅允许在Windows系统运行),再而为完成核电站安全分析时需要大量抽样,因此单台计算机运行该软件显现出运行时间久、效率低等缺点。为减少运行时间,提高该软件的效率,需要搭建Windows系统分布式集群环境,来弥补这些缺点。
\end{abstract}